\documentclass[journal,12pt,twocolumn]{IEEEtran}

\usepackage{setspace}
\usepackage{gensymb}
\singlespacing
\usepackage[cmex10]{amsmath}
\usepackage{amssymb}
\usepackage{xurl}
\usepackage{tabularx}
\usepackage{amsthm}
\usepackage{comment}
\usepackage{mathrsfs}
\usepackage{txfonts}
\usepackage{stfloats}
\usepackage{bm}
\usepackage{cite}
\usepackage{cases}
\usepackage{subfig}

\usepackage{longtable}
\usepackage{multirow}
\usepackage{cancel}
\usepackage{enumitem}
\usepackage{mathtools}
\usepackage{steinmetz}
\usepackage{tikz}
\usepackage{circuitikz}
\usepackage{verbatim}
\usepackage{tfrupee}
\usepackage[breaklinks=true]{hyperref}
\usepackage{graphicx}
\usepackage{tkz-euclide}

\usetikzlibrary{calc,math}
\usepackage{listings}
    \usepackage{color}                                            %%
    \usepackage{array}                                            %%
    \usepackage{longtable}                                        %%
    \usepackage{calc}                                             %%
    \usepackage{multirow}                                         %%
    \usepackage{hhline}                                           %%
    \usepackage{ifthen}                                           %%
    \usepackage{lscape}     
\usepackage{multicol}
\usepackage{chngcntr}

\DeclareMathOperator*{\Res}{Res}

\renewcommand\thesection{\arabic{section}}
\renewcommand\thesubsection{\thesection.\arabic{subsection}}
\renewcommand\thesubsubsection{\thesubsection.\arabic{subsubsection}}

\renewcommand\thesectiondis{\arabic{section}}
\renewcommand\thesubsectiondis{\thesectiondis.\arabic{subsection}}
\renewcommand\thesubsubsectiondis{\thesubsectiondis.\arabic{subsubsection}}


\hyphenation{op-tical net-works semi-conduc-tor}
\def\inputGnumericTable{}                                 %%

\lstset{
%language=C,
frame=single, 
breaklines=true,
columns=fullflexible
}
\begin{document}


\newtheorem{theorem}{Theorem}[section]
\newtheorem{problem}{Problem}
\newtheorem{proposition}{Proposition}[section]
\newtheorem{lemma}{Lemma}[section]
\newtheorem{corollary}[theorem]{Corollary}
\newtheorem{example}{Example}[section]
\newtheorem{definition}[problem]{Definition}

\newcommand{\BEQA}{\begin{eqnarray}}
\newcommand{\EEQA}{\end{eqnarray}}
\newcommand{\define}{\stackrel{\triangle}{=}}
\bibliographystyle{IEEEtran}
\raggedbottom
\setlength{\parindent}{0pt}
\providecommand{\mbf}{\mathbf}
\providecommand{\pr}[1]{\ensuremath{\Pr\left(#1\right)}}
\providecommand{\qfunc}[1]{\ensuremath{Q\left(#1\right)}}
\providecommand{\sbrak}[1]{\ensuremath{{}\left[#1\right]}}
\providecommand{\lsbrak}[1]{\ensuremath{{}\left[#1\right.}}
\providecommand{\rsbrak}[1]{\ensuremath{{}\left.#1\right]}}
\providecommand{\brak}[1]{\ensuremath{\left(#1\right)}}
\providecommand{\lbrak}[1]{\ensuremath{\left(#1\right.}}
\providecommand{\rbrak}[1]{\ensuremath{\left.#1\right)}}
\providecommand{\cbrak}[1]{\ensuremath{\left\{#1\right\}}}
\providecommand{\lcbrak}[1]{\ensuremath{\left\{#1\right.}}
\providecommand{\rcbrak}[1]{\ensuremath{\left.#1\right\}}}
\theoremstyle{remark}
\newtheorem{rem}{Remark}
\newcommand{\sgn}{\mathop{\mathrm{sgn}}}
\providecommand{\abs}[1]{\vert#1\vert}
\providecommand{\res}[1]{\Res\displaylimits_{#1}} 
\providecommand{\norm}[1]{\lVert#1\rVert}
%\providecommand{\norm}[1]{\lVert#1\rVert}
\providecommand{\mtx}[1]{\mathbf{#1}}
\providecommand{\mean}[1]{E[ #1 ]}
\providecommand{\fourier}{\overset{\mathcal{F}}{ \rightleftharpoons}}
%\providecommand{\hilbert}{\overset{\mathcal{H}}{ \rightleftharpoons}}
\providecommand{\system}{\overset{\mathcal{H}}{ \longleftrightarrow}}
	%\newcommand{\solution}[2]{\textbf{Solution:}{#1}}
\newcommand{\solution}{\noindent \textbf{Solution: }}
\newcommand{\cosec}{\,\text{cosec}\,}
\providecommand{\dec}[2]{\ensuremath{\overset{#1}{\underset{#2}{\gtrless}}}}
\newcommand{\myvec}[1]{\ensuremath{\begin{pmatrix}#1\end{pmatrix}}}
\newcommand{\mydet}[1]{\ensuremath{\begin{vmatrix}#1\end{vmatrix}}}
\newcommand*{\permcomb}[4][0mu]{{{}^{#3}\mkern#1#2_{#4}}}
\newcommand*{\perm}[1][-3mu]{\permcomb[#1]{P}}
\newcommand*{\comb}[1][-1mu]{\permcomb[#1]{C}}
\numberwithin{equation}{subsection}
\makeatletter
\@addtoreset{figure}{problem}
\makeatother
\let\StandardTheFigure\thefigure
\let\vec\mathbf
\renewcommand{\thefigure}{\theproblem}
\def\putbox#1#2#3{\makebox[0in][l]{\makebox[#1][l]{}\raisebox{\baselineskip}[0in][0in]{\raisebox{#2}[0in][0in]{#3}}}}
     \def\rightbox#1{\makebox[0in][r]{#1}}
     \def\centbox#1{\makebox[0in]{#1}}
     \def\topbox#1{\raisebox{-\baselineskip}[0in][0in]{#1}}
     \def\midbox#1{\raisebox{-0.5\baselineskip}[0in][0in]{#1}}
\vspace{3cm}
\title{AI1103 : Assignment 5}
\author{Revanth badavathu - CS20BTECH11007}
\maketitle
\newpage
\bigskip
\renewcommand{\thefigure}{\arabic{figure}}
\renewcommand{\thetable}{\arabic{table}}

 Download latex-tikz codes from 
%
\begin{lstlisting}
https://github.com/cs20btech11007/Assignment-5/blob/main/Assignment%205.tex
\end{lstlisting}
\section*{\textbf{Problem-(CSIR UGC NET EXAM) (Dec-2014),Q-116}}
Let Y follows multivariate normal distribution $N_{n}(0,I)$ and let A and B be a $n\times n$ symmetric,idempotent matrices. Then which of the following statements are true? \\
\\1.if AB=0 ,then Y'AY and Y'BY are independently distributed.\\
\\2.if Y'(A+B)Y has chi square distribution then Y'AY, Y'BY are independently distributed.\\
\\3.Y'(A-B)Y has chi square distribution.\\
\\4.Y'AY, Y'BY has chi square distribution.\\
\section*{Solution}
\begin{lemma}
 $Y=(Y_1,Y_2,Y_3.....,Y_n)$ has multivariate normal distribution with mean $\mu=0$ and variance-covariance matrix $\sum=I$.\\
The distributions Y'$A$Y and  Y'$B$Y  are independently distributed if and only  $A\sum B$=0.\\
 \end{lemma}
\begin{lemma}
Let {Y \thicksim N_k($\mu$, \sum)},  $\sum$ $>$ 0 and A be a  $n\times n$ real symmetric matrix.\\
Then $Y'AY \thicksim \chi^2_m(\lambda)$  with  $\lambda=\frac{1}{2}\mu'A\mu$ if and only if $A\sum$ is idempotent of rank m.\\
\end{lemma}
We are proving this by proof by using moment-generating function of $\chi^2_m(\lambda)$
\begin{align}
    M_{\chi_m^2(\lambda)}=(1-2t)^{\frac{-m}{2}}e^{\frac{2t\lambda}{1-2t}}, t< \frac{1}{2}
\end{align}
\begin{proof}
\\ \textbf{Sufficiency.}Suppose $(A\sum)^2=A\sum$ and $r(A \textstyle\sum)=m$, where $\sum= KK'$
and $A_0= K'AK$. Then $A_0^2$ =$A_0$ and r($A_0$) = m. Thus, there exists an orthogonal
matrix P such that\\
\begin{align}
    A_0=P\begin{pmatrix}I_m & 0\\ 0 & 0 \end{pmatrix}P'
\end{align}
\\where P = ($P_1$, $P_2$), $P_1P_1 = I_m$ and z = $P_1'K^{-1}Y$. It follows that\\
\begin{align}
Y'AY = z'z \thicksim \chi_m^2(\lambda),
\end{align}
where $\lambda =\frac{1}{2}(P_1'K^{-1}Y)'(P_1'K^{-1}Y)$ =$\frac{1}{2}\mu\mu'$\\
\\\textbf{Necessity}. Suppose $Y'AY \thicksim \chi_m^2(\lambda)$. Let $P = (P_1,....., P_k)$ be an orthogonal
matrix such that $P'A_0P =\Lambda = diag(\lambda_1,..., \lambda_k)$, where $\lambda_1 $\geq$ ... $\geq$\lambda_k$ are eigenvalues of $A_0$. Then\\
\begin{align}
Y'AY &=z'\Lambda z =\sum_{i=1}^{k} \lambda_i z_i^2 
\end{align}
\begin{align} M_{Y'AY}(t)=\prod\limits_{i=1}^{k}M_{z_i^2}(t\lambda_i)
\end{align}
where $z = P'K^{-1}Y \thicksim N_k(P'K^{-1}\mu ,I)$ and  $z_1, ..., z_k$ are independent. Hence,\\
\begin{align}
(1-2t)^{\frac{-m}{2}}e^{\frac{2t\lambda}{1-2t}}=\prod\limits_{i=1}^{k}(1-2t\lambda_i)^{\frac{-1}{2}}e^{\frac{t\lambda_i}{1-2t\lambda_i}(p_i'K^{-1}\mu)}
\end{align}
for t < 1/2 and $t \lambda_i <\frac{1}{2} (i = 1, ..., k)$. Comparing the discontinuous points of
the two functions on both sides results in\\
\begin{align}
    (1-2t)^{\frac{-m}{2}}=\prod\limits_{i=1}^{k}(1-2t\lambda_i)^{\frac{-1}{2}}=|I_k-2t A_0|^{\frac{-1}{2}}
\end{align}
which implies that $\lambda_1 = ... = \lambda_m = 1$ and $\lambda_{m+1}= ....=\lambda_{k}$ = 0.\\
Thus, $A_0$ or $A\sum$ is idempotent of rank m.\\
\textbf{The proof is completed.}\\
\end{proof}
\\\textbf{Examining each option :} 
\begin{enumerate}
\item
If Y'AY and Y'BY are independently distributed then $A\sum B=0$.\\
we know that Y is a multi normalvariet distribution so the $\sum$ is equal to I then, $AB$ must be equal to 0.\\
so that the distributions  Y'AY and Y'BY are independently distributed.\\
\\Hence, \textbf{Option 1 is correct.}\\
\item
Given that Y'(A+B)Y has chi square distribution then the matrix $\sum(A+B)$ must be an idempotent matrix.\\
  \begin{align}
 (\textstyle \sum(A+B))^2=\sum(A+B) 
  \end{align}
   We know that $\sum=I$ (identity matrix)
    \begin{align}
  A^2+B^2+AB+BA=A+B 
  \end{align}
  \\Given that A and B are idempotent matrices.
    \begin{align}
 A^2=A ,B^2=B  
  \end{align}
   \begin{align}
 \cancel{A}+\cancel{B}+AB+BA=\cancel{A}+\cancel{B}
  \end{align}
  \begin{align}
 AB+BA=0 
  \end{align}
 Then AB,BA must be equal to 0,\\
The distributions Y'$A_1$Y and  Y'$B$Y  are independently distributed if $A\sum B$=0.\\
 \\We know that $\sum=I$ (identity matrix)
 \begin{align}
 A \textstyle \sum B=AB=0
   \end{align}
  \\ clearly we can say that the distributions Y'$A$Y and  Y'$B$Y  are independently distributed .\\
   \\Hence, \textbf{Option 2 is correct.}\\
   \item
 \\ For having chi square distribution for a normally distributed Y'(A-B)Y distribution .
 \\Then the matrix $(\sum(A-B)$  must be an idempotent matrix.\\
  \begin{align}
 (\textstyle \sum(A-B))^2=\sum(A-B) 
  \end{align}
   We know that $\sum=I$ (identity matrix)
    \begin{align}
 A^2+B^2-AB-BA=A-B 
  \end{align}
  \\we know that A and B are idempotent matrices.
    \begin{align}
 A^2=A ,B^2=B 
  \end{align}
   \begin{align}
 \cancel{A}+{B}-AB-BA=\cancel{A}+{-B}
  \end{align}
  \begin{align}
 AB+BA=2B
  \end{align}
  \\for having chi square distribution the above equation(11) need to be satisfied.\\
  but it is not mentioned in the option 3.\\
  so,we cannot say that the distribution $Y'(A-B)Y$ has chi square distribution.\\
   \\ Hence, \textbf{Option 3 is incorrect.}\\
 \\4.) $Y=(Y_1,Y_2,Y_3.....,Y_n)$ has multivariate normal distribution with mean $\mu=0$ and variance-covariance matrix $\sum=I$.\\
 Then Y'AY and Y'BY has chi square distribution($\chi^2$)  if and only if $\sum A$ and $\sum B$ are idempotent matrix.\\
 \\we know that A and B are idempotent matrices.
  \begin{align}
 A^2=A ,B^2=B  
  \end{align}
  We know that $\sum=I$ (identity matrix)
  \begin{align}
 (\textstyle \sum  A)^2=A ,(\textstyle \sum B)^2=B 
  \end{align}
 Then $Y'AY$ and $Y'BY$ has chi square distribution($\chi^2$).\\
\\Hence, \textbf{Option 4 is correct.}
 \end{enumerate}
\end{document}
